\section{Conclusion}

  This paper presents Lever, a pre-scheduling straggler mitigation framework that exploits the predictability of recurring batch stream jobs to optimize the assignment of data. Lever identifies potential stragglers by analyzing execution information of historical jobs and introduces Iterative Learning Control(ILC) model to evaluate nodes' capability. Furthermore, Lever carefully chooses helpers and optimizes the assignment of the job input data according to each node's capability proportion. Lever has been implemented on the top of Spark Streaming. Lever is also open source and is now included in Spark Packages at \url{http://spark-packages.org/package/trueyao/spark-lever}. We conduct various experiments to validate the effectiveness of Lever. The experimental results demonstrate that Lever reduces job completion time by 30.72\% to 42.19\% and outperforms traditional techniques significantly.

  In future work, we plan to enhance the accuracy for estimation of node's computational capability. A possible approach is to introduce a new machine learning method to evaluate each node by collecting the node-level features of hardware configuration and resource utilization such as CPU, memory, disk and network. By using a statistical model to analyze these information, we can get a relatively more accurate result to direct capability-aware pre-scheduling. Another direction in the future is that we will consider the influence of shuffle-heavy tasks and how to balance load at shuffle stage. It will extend the usage of Lever.

