\section{Related Work}

  Our work is related to the research in batch stream processing and straggler mitigation on heterogeneous clusters. We briefly discuss the most related work.

  \textbf{Batch Stream Processing and Incremental Data Processing Systems.} Batch stream processing systems collect received data into batches and periodically process them using MapReduce-style batch computations. The typical systems include Comet \cite{He2010} which is structured on DryadLINQ, HOP \cite{Condie2010} which leverages the power of batch framework MapReduce, and Spark Streaming \cite{Zaharia2013} which is built on top of Spark. These systems take full advantage of characteristics in batch processing engine such as high throughput and fault-tolerance. Some other systems \cite{Li2011} and \cite{Lam2012} intent to modify batch framework to adapt to the requirements of stream processing. Other stream processing systems such as Borealis \cite{Abadi2005}, TimeStream \cite{Qian2013}, Naiad \cite{Murray2013}, Storm \cite{Toshniwal2014} and Heron \cite{Kulkarni2015} are based on the \emph{continuous operator model}. In this model, the streaming computation is expressed as a graph of long-lived operators that exchange messages with each other to process the streaming data.Incremental bulk processing systems such CBP \cite{Logothetis2010},Percolator \cite{Peng2010} and Incoop \cite{Bhatotia2011a} allows updated view of processed data set to be maintained by incrementally and efficiently recomputing the updates to the input data. In such systems, the recomputation is triggered whenever an update to the input dataset is detected.

  \textbf{Straggler and Data Skew on Heterogeneous Clusters.} Load imbalance on heterogeneous clusters are common phenomenon in cluster computing environments because of heterogeneity, straggler, data skew and so on. Many techniques have been presented to solve these problems. The typical scheduling methods are Delay Scheduling \cite{Zaharia2010B}, LATE \cite{Zaharia2008}, and Tarazu \cite{Ahmad2012}. Delay Scheduling tries to maintain data locality by later decision. LATE speculates slow tasks using a notion of progress scores. Tarazu designs communication-aware load balancing of map computation, communication-aware scheduling of map computation, and predictive load balancing of reduce computation to respectively prevent shuffle-critical tasks stealing, interleave remote tasks with local ones, and skew the intermediate key distribution among the reduce tasks. They are reactive ways for batch processing. The typical straggler mitigation approaches are Speculative Execution \cite{Dean2004}, Mantri \cite{Ananthanarayanan2010}, Dolly \cite{Ananthanarayanan2013}, GRASS \cite{Ananthanarayanan2014} and Wrangler \cite{Yadwadkar2014}. Speculative Execution marks slow tasks as stragglers and launch a redundant copy of a task-in-progress on a different node. Using real-time progress reports, Mantri monitors, detects and acts on outliers by restarting outliers, network-aware placement of tasks and protecting outputs of valuable tasks. Dolly is a replication-based method, it proposes full cloning of small jobs by delay assignment. Dolly don't need to wait to observe before acting with a proactive approach of cloning jobs. But it incurs extra resources. GRASS is designed for approximation jobs, it delicately balances immediacy of improving the approximation goal with the long term implications of using extra resources for speculation. Wrangler automatically learns to predict stragglers using a statistical learning technique based on cluster resource utilization counters. It is a straggler-avoid method.The typical skew mitigation techniques are Scarlett \cite{Ananthanarayanan2011}, SkewTune \cite{Kwon2012}. Scarlett replicates block based on their popularity by accurately predicting file popularity and working within hard bounds on additional storage.SkewTune solves the problem of load balancing in MapReduce-like systems by identifying the task with the greatest expected remaining processing time and redistributing the unprocessed data from the stragglers to other workers.

  \textbf{Load Balancing for Stream Processing.} Many load balancing techniques for stream processing \cite{Shah2003}, \cite{Cherniack2003}, \cite{Xing2005}, \cite{Anis2015} have been proposed. Flux \cite{Shah2003} monitors the load of each operator, ranks servers by load, and migrates operators from the most loaded to the least loaded server, from the second most loaded to the second least loaded, and so on. Aurora and Medusa propose policies to migrating operators in stream processing \cite{Cherniack2003}. Borealis uses a similar approach but it also aims at reducing the correlation of load spikes among operators placed on the same server \cite{Xing2005}. \cite{Anis2015} introduces a new stream partitioning scheme that adapts the classical power of two choices to a distributed streaming setting by leveraging two techniques: key splitting and local load estimation. These techniques are designed for continuous operator stream processing, not suitable for batch stream processing.